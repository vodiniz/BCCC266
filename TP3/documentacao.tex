\documentclass{article}

% \usepackage[english]{babel}
\usepackage[brazilian]{babel}
\usepackage[letterpaper,top=2cm,bottom=2cm,left=3cm,right=3cm,marginparwidth=1.75cm]{geometry}

% Useful packages
\usepackage{amsmath}
\usepackage{graphicx}
% \usepackage[colorlinks=true,linkcolor=blue]{hyperref}
\usepackage{hyperref}
\usepackage{listings}
\usepackage[figure,table,lstlisting]{totalcount}
\usepackage{tcolorbox}
\usepackage{listings}
\usepackage{xcolor}

\hypersetup{
    colorlinks=true,
    linkcolor=black,
    filecolor=magenta,      
    urlcolor=cyan,
    pdftitle={Overleaf Example},
    pdfpagemode=FullScreen,
    }

\definecolor{verde}{rgb}{0.25,0.5,0.35}
\definecolor{jpurple}{rgb}{0.5,0,0.35}

\lstset{
  language=C++,
  basicstyle=\ttfamily\small,
  keywordstyle=\color{jpurple}\bfseries,
  stringstyle=\color{red},
  commentstyle=\color{verde},
  morecomment=[s][\color{blue}]{/**}{*/},
  extendedchars=true,
  showspaces=false,
  showstringspaces=false,
  numbers=left,
  numberstyle=\tiny,
  breaklines=true,
  backgroundcolor=\color{cyan!10},
  breakautoindent=true,
  captionpos=b,
  xleftmargin=0pt,
  tabsize=4
}

\renewcommand{\lstlistingname}{Código}
\renewcommand{\lstlistlistingname}{Lista de Códigos Fonte}

\newcommand\conditionalLoF{%
    \iftotalfigures
        \listoffigures
    \fi}
\newcommand\conditionalLoT{%
    \iftotaltables
        \listoftables
    \fi}
\newcommand\conditionalLoL{%
    \iftotallstlistings
        \lstlistoflistings
    \fi}

\newcommand{\DESCRICAO}[1]{
    % set flexible interword space
    \setlength{\spaceskip}{0.5em plus 1em minus 0.1em}%
    % add some space with not as much flexibility, but only
    % if some space precedes
    \ifdim\lastskip>0pt \hspace{0.5em plus 0.5em minus 0.1em}\fi
    \texttt{\textbf{\color{red}#1}}
}

\newcommand{\CAPA}[6]{
    % 1) Título
    % 2)
    % 3)
    % 4)
    \begin{titlepage}
    	
    	  \vfill
    	
    	  \begin{center}
    	    \begin{large}
    	      Universidade Federal de Ouro Preto - UFOP
    	    \end{large}
    	  \end{center}
    	
    	  \begin{center}
    	    \begin{large}
    	      Instituto de Ciências Exatas e Biológicas - ICEB
    	    \end{large}
    	  \end{center}
    	
    	  \begin{center}
    	    \begin{large}
    	      Departamento de Computação - DECOM
    	    \end{large}
    	  \end{center}
          
          \begin{center}
    	    \begin{large}
              Ciência da Computação
    	    \end{large}
    	  \end{center}
    	  
    	  \vfill
          \vfill
    	
    	  \begin{center}
    	    \begin{Huge}
    	      #1\\
    	    \end{Huge}
    	    \begin{Large}
              #2\\
    	    \end{Large}
    	  \end{center}
    	  
    	  \vfill
    	  
    	  \begin{center}
    	    \begin{large}
              #3
    	    \end{large}
    	  \end{center}

        \begin{center}
    	    \begin{large}
              #4
    	    \end{large}
    	  \end{center}

        \begin{center}
    	    \begin{large}
              #5
    	    \end{large}
    	  \end{center}
    	
    	  \begin{center}
    	    \begin{large}
    	      Professor: #6
    	    \end{large}
    	  \end{center}
    	
    	  \vfill
          \vfill
    	
    	  \begin{center}
    	    \begin{large}
    	      Ouro Preto \\
    	      \today \\
    	    \end{large}
    	  \end{center}    
    
    
    \clearpage
    \newpage

    \tableofcontents
    \conditionalLoF
    \conditionalLoT
    \conditionalLoL
	
    % \thispagestyle{empty}
    % \tableofcontents
    % \newpage
    % \thispagestyle{empty}
    
    % \listoffigures
    
    % \lstlistoflistings
    % \newpage
    % \thispagestyle{empty}
    \end{titlepage}
}

\begin{document}

\CAPA{Trabalho Prático III}{BCC266 - Organização de Computadores}{Vitor Oliveira Diniz}{Jéssica Machado}{}{Pedro Silva}



\section{Introdução}
O HD  é um dispositivo de armazenamento de dados que armazena e recupera dados digitais usando armazenamento magnético. Os dados são acessados de maneira aleatória, o que significa wue blocos individuais de de dados que podem ser armazenados e recuperados em qualquer ordem, além de
ser um armazenamento que retém os dados mesmo quando a máquina é desligada.
\subsection{Especificações do problema}

Para este trabalho prático, deveríamos, a partir do TP2 que já realizado, implementar o HD se caso palavra de um bloco de memória não for encontrada na RAM, portanto a memória
externa vai ser considerada como último nível. Além disso deveriamos adicionar um tratador de interrupções, que verificaria se
há alguma interrupção a ser tratada e deve parar a execução do código atual para tratar a mesma. 

\subsection{Considerações Iniciais}
Algumas ferramentas foram utilizadas durante a criação deste projeto:

\begin{itemize}
  \item Ambiente de desenvolvimento do código fonte: Visual Studio Code.
  \item Linguagem utilizada: C.
  \item Ambiente de desenvolvimento da documentação: Visual Studio Code \LaTeX Workshop.
\end{itemize}

\subsection{Ferramentas utilizadas}
Algumas ferramentas foram utilizadas para testar a implementação, como:

\begin{itemize}
    \item[-] \textit{CLANG}: ferramentas de análise estática do código.
    \item[-] \textit{Valgrind}: ferramentas de análise dinâmica do código.
\end{itemize}

\subsection{Especificações da máquina}
A máquina onde o desenvolvimento e os testes foram realizados possui a seguinte configuração:

\begin{itemize}
    \item[-] Processador: Ryzen 7-5800H.
    \item[-] Memória RAM: 16 Gb.
    \item[-] Sistema Operacional: Arch Linux x86\_64.
\end{itemize}


\subsection{Instruções de compilação e execução}

Para a compilação do projeto, basta digitar:

\begin{tcolorbox}[title=Compilando o projeto,width=\linewidth]
    gcc main.c -c -Wall \newline
    gcc mmu.c -c -Wall \newline
    gcc memory.c -c -Wall\newline
    gcc instruction.c -c -Wall\newline
    gcc cpu.c -c -Wallnewline
    gcc generator.c -c -Wall\newline
    gcc main.o mmu.o memory.o instruction.o cpu.o generator.o -o exe -g
\end{tcolorbox}

Usou-se para a compilação as seguintes opções:

\begin{itemize}
    \item [-] \emph{-Wall}: para mostrar todos os possível \emph{warnings} do código.
    \item [-] \emph{-c}: Para compilar o arquivo sem linkar os arquivos para obtermos um arquivo do tipo objeto.
    \item [-] \emph{-o}: Compilar para um arquivo do tipo output $($saída$)$.
\end{itemize}

Para a execução do programa basta digitar um dos exemplos:
\begin{tcolorbox}[title=,width=\linewidth]
    ./exe random [TAMANHO DA RAM] [L1] [L2] [L3] [TAMANHO DO HD]\newline


\end{tcolorbox}


\clearpage



\section{Desenvolvimento}

Seguindo as boas práticas de programação, implementamos o HD particionada em blocos e simulamos o mapeamento associativo para a troca das mesmas com a memória RAM. Usamos as políticas LFU (Last Frequently Used) e LRU (Last Recently Used) como métodos de otimização.



\subsection{Operações}

A seguir entraremos em detalhe sobre as principais funções utilizadas no programa e as coisas que implementamos.

\subsubsection{Médodos de mapeamento}

Define o tipo de método que vai ser usado para a saída dos resultados.

\begin{lstlisting}[caption={Definição do tipo de método},label={lst:cod1},language=C]

    // 1 MAPEAMENTO DIRETO
    // 2  LRU (Least Recently Used)
    // 3  LFU (Least Frequently Used)

    #define SUBSTITUTION_METHOD 3
    
\end{lstlisting}

\subsubsection{TAD File e TAD HD}

Criamos um TAD File que salva o nome do arquivo na variável fileName, e uma variável file do tipo FILE. E no TAD HD criamos o int size para receber o tamanho do hd
e um vetor de Files.

\begin{lstlisting}[caption={TAD Machine},label={lst:cod2},language=C]

typedef struct {
    char fileName[FILE_STR_TAM]; //salvar o nome do arquivo
    FILE *file;    
} File;
    
typedef struct {
    int size;
    File *files; //arquivos
} HD;

\end{lstlisting}

\subsubsection{TAD Machine}

Adicionamos o HD hd, hitRAM e o hitHd (que conta quantas vezes processador achou o dado na ram e no HD) e o missRam (que conta quantas vezes o processador precisou buscar o dado no HD).
O int interruption foi criado para garantir que haja apenas uma interrupção ocorrendo, para que não tenhamos uma recursão infinita na nossa máquina.

\begin{lstlisting}[caption={TAD Machine},label={lst:cod2},language=C]

typedef struct {
    Instruction* instructions;
    RAM ram;
    Cache l1; // cache L1
    Cache l2; // cache L2
    Cache l3; // cache L3
    HD hd;
    int hitL1, hitL2, hitL3, hitRAM, hitHD;
    int missL1, missL2, missL3, missRAM;
    int totalCost;
    int interruption;
} Machine;

\end{lstlisting}

\subsubsection{startHD e stopHD}

Na função startHD modificamos a variável size que está no hd-$>$size para o size passado por parâmetro e alocamos dinamicamente o vetor de arquivos.
Na estrutura de repetição for, criamos os arquivos e inicializamos eles com valores aleatórios.
Na função stopHD simplesmente liberamos o vetor de arquivos que estava presente no hd.

\begin{lstlisting}[caption={Funçao run},label={lst:cod3},language=C]
void startHD(HD* hd, int size){
    
    hd->size = size;
    hd->files = malloc (size * sizeof(File));

    for(int i = 0; i < size; i++){
        sprintf(hd->files[i].fileName, "0x0%d.txt", i);
        // strcpy(, nomeArquivo);
        hd->files[i].file = fopen(hd->files[i].fileName, "w");

        fprintf(hd->files[i].file, "%d\n", WORDS_SIZE);


        for (int j=0;j<WORDS_SIZE;j++){

            fprintf(hd->files[i].file, "%d\n", rand() % 100);

        }

        fclose(hd->files[i].file);
    }

}

void stopHD(HD* hd){
    
    free(hd->files);

} 
        
\end{lstlisting}

\subsubsection{start}

Nessa função, iniciamos o valor do hitRam, hitHD e missRAM como 0. Além de adicionarmos o a função startHD.

\begin{lstlisting}[caption={Funçao run},label={lst:cod3},language=C]
    void start(Machine* machine, Instruction* instructions, int* memoriesSize) {
        startRAM(&machine->ram, memoriesSize[0]);
        startCache(&machine->l1, memoriesSize[1]);
        startCache(&machine->l2, memoriesSize[2]);
        startCache(&machine->l3, memoriesSize[3]);
        startHD(&machine->hd, memoriesSize[4]);
        machine->interruption = 0;
    
        machine->instructions = instructions;
        machine->hitL1 = 0;
        machine->hitL2 = 0;
        machine->hitL3 = 0;
        machine->hitRAM = 0;
        machine->hitHD = 0;
    
        machine->missL1 = 0;
        machine->missL2 = 0;
        machine->missL3 = 0;
        machine->hitRAM = 0;
        machine->totalCost = 0;
    }
        
        
\end{lstlisting}

\subsubsection{run}

Inicializa o contador PC com valor 0 e, enquanto o opcode de machine-$>$instructions[PC] for diferente de -1 (condição de parada), printamos a quantidade de vezes que o processador achou (hit) o dado na L1, L2, L3, RAM e no HD, a quantidade de vezes que o processador precisou buscar o dado no HD (miss), e o custo total.

\begin{lstlisting}[caption={Funçao run},label={lst:cod3},language=C]

void run(Machine* machine) {    
    int PC = 0; // Program Counter
    while(machine->instructions[PC].opcode != -1) {
        executeInstruction(machine, PC++);
        printf("\tL1:(%6d, %6d) | L2:(%6d, %6d) | L3:(%6d, %6d) | RAM:(%6d, %6d) | HD:(%6d) | COST: %d\n", 
            machine->hitL1, machine->missL1, 
            machine->hitL2, machine->missL2,
            machine->hitL3, machine->missL3,
            machine->hitRAM, machine->missRAM,
            machine->hitHD,
             machine->totalCost);
    }
}
    
    
\end{lstlisting}

\subsubsection{printMemories}

Aqui adicionamos uma linha a mais na tabela para representar cada bloco presente no HD.
Printa as informações da RAM, se está atualizado é a cor verde e, se não, cor vermelha.

\begin{lstlisting}[caption={Função printMemories},label={lst:cod4},language=C]
    void printMemories(Machine* machine) {
    printf("\x1b[0;30;47m     ");
    printc("HD", WORDS_SIZE * 8 - 1);
    printc("RAM", WORDS_SIZE * 8 - 1);
    printc("Cache L3", WORDS_SIZE * 8 + 6);
    printc("Cache L2", WORDS_SIZE * 8 + 6);
    printc("Cache L1", WORDS_SIZE * 8 + 6);
    printf("\x1b[0m\n");

    MemoryBlock hd_block;

    for (int i=0;i<machine->hd.size;i++) {
        printf("\x1b[0;30;47m%5d|\x1b[0m", i);
        hd_block = readHDblocks(&machine->hd.files[i]);

        for (int j=0;j<WORDS_SIZE;j++)
            printf(" %5d |", hd_block.words[j]);


        if(i < machine->ram.size){
            printf("|");
            for (int j=0;j<WORDS_SIZE;j++)
                printf(" %5d |", machine->ram.blocks[i].words[j]);

            if (i < machine->l3.size) {
                printf("|");
                printcolored(machine->l3.lines[i].tag, machine->l3.lines[i].updated);
                for (int j=0;j<WORDS_SIZE;j++)
                        printf(" %5d |", machine->l3.lines[i].block.words[j]);

                if (i < machine->l2.size) {
                    printf("|");
                    printcolored(machine->l2.lines[i].tag, machine->l2.lines[i].updated);
                    for (int j=0;j<WORDS_SIZE;j++)
                        printf(" %5d |", machine->l2.lines[i].block.words[j]);
                    if (i < machine->l1.size) {
                        printf("|");
                        printcolored(machine->l1.lines[i].tag, machine->l1.lines[i].updated);
                        for (int j=0;j<WORDS_SIZE;j++)
                            printf(" %5d |", machine->l1.lines[i].block.words[j]);
                    }
                }
            }

        }

        printf("\n");
    }


}
    

\end{lstlisting}
\clearpage
\subsubsection{readHDblocks}

Lê os blocos de HD que estão salvos em um arquivo.txt e retorna uma palavra do tamanho definido no programa.

\begin{lstlisting}[caption={startCache},label={lst:cod6},language=C]

MemoryBlock readHDblocks(File *hd){

    MemoryBlock block;

    hd->file = fopen(hd->fileName, "r");

    int words_size;
    fscanf(hd->file, "%d", &words_size);

    for(int i = 0; i < WORDS_SIZE; i++){
        fscanf(hd->file, "%d", &block.words[i]);
    }

    fclose(hd->file);
    return block;

}

\end{lstlisting}

\subsubsection{memoryCacheMapping}

Ela é responsável por checar qual tipo de mapeamento estamos utilizando e, dependendo do mapeamento,
iremos aplicar uma política diferente. No caso do mapeamento direto, retornamos o possível endeeço da cache e a verificação
de sua existência é feita no MMUSearchOnMemorys. No caso da LRU, percorremos o vetor sempre checando qual endereço está a mais tempo na cache
e salvando seu índice, caso encontremos o endereço que procuramos retornamos seu índice, se a substituição vai acontecer ou não depende do MMUSearchOnMemorys.
No LFU acontece a mesma coisa que o LRU, apenas mudando a condição de checagem para o bloco menos usado. 

\begin{lstlisting}[caption={Função memoryCacheMapping},label={lst:cod6},language=C]

    int memoryCacheMapping(int address, Cache* cache) {

    int index = 0;

    switch(SUBSTITUTION_METHOD){
        //DIRECT MAPPING
        case 1:
            return address % cache->size;
            break;
        //LRU METHOD (Least Recently Used)
        case 2:
            cache->lines[0].timeOnCache++;
            for( int i = 0; i < cache->size; i++){
                
                cache->lines[i].timeOnCache++;

                if(cache->lines[i].timeOnCache > cache->lines[index].timeOnCache)
                    index = i;
                
                if(cache->lines[i].tag == address)
                    return i;

            }
            return index;
            break;

        //LFU METHOD (Least Frequently Used)
        case 3:
            for( int i = 0; i < cache->size; i++){
                if(cache->lines[i].timesUsed < cache->lines[index].timesUsed)
                    index = i;

                if(cache->lines[i].tag == address)
                    return i;
            }

            return index;
            break;
    }

    return address % cache->size;
}

\end{lstlisting}

\subsubsection{updateMachineInfos}

Modificamos o valor do hitHD e do missRAM. Fazemos a alteração do custo total da máquina, tempo na cache e quantas vezes que foi usada.

\begin{lstlisting}[caption={Função updateMachineInfos},label={lst:cod6},language=C]

void updateMachineInfos(Machine* machine, Line* line) {

    switch (line->cacheHit) {
        case 1:
            machine->hitL1 += 1;
            break;

        case 2:
            machine->hitL2 += 1;
            machine->missL1 += 1;
            break;
        
        case 3:
            machine->hitL3 += 1;
            machine->missL1 += 1;
            machine->missL2 += 1;
            break;
        
        case 4:
            machine->hitRAM += 1;
            machine->missL1 += 1;
            machine->missL2 += 1;
            machine->missL3 += 1;
            break;

        case 5:
            machine->hitHD += 1;
            machine->missRAM += 1;
            machine->missL1 += 1;
            machine->missL2 += 1;
            machine->missL3 += 1;
            break;
    }

    line->timeOnCache = 0;
    line->timesUsed++;

    machine->totalCost += line->cost;
}  

\end{lstlisting}
\clearpage

\subsubsection{MMUSearchOnMemorys}

Inicialmente, essa função pega a possível posição do bloco que queremos na CACHE e conferimos se o
que queremos está realmente na RAM. Se estiver, atualizamos o cache hit referente aonde ele se encontra,
o custo de acesso da memória, a nossa máquina e retornamos a linha que o endereço se encontra. Se não encontrarmos o 
endereço na cache, fazemos as movimentações necessárias entre as memórias, levando em consideração o mapeamento
escolhido e, caso precise, levamos o endereço da cache para RAM. Como proposto para o trabalho prático deveriamos ter adicionado
uma verificação a mais. Caso o bloco desejado não esteja presente na RAM, deveriamos pegar essa informação do HD, que no nosso caso 
é simulado por diferentes arquivos do tipo txt, cada um com o nome referente ao bloco. Depois de pegarmos o valor no HD, deveriamos 
salvar ele na cache, sempre obedecendo o mapeamento escolhido.


\begin{lstlisting}[caption={Função MMUSearchOnMemorys},label={lst:cod6},language=C]

    Line* MMUSearchOnMemorys(Address add, Machine* machine) {
        // Strategy => write back
        
        int l1pos = memoryCacheMapping(add.block, &machine->l1);
        int l2pos = memoryCacheMapping(add.block, &machine->l2);
        int l3pos = memoryCacheMapping(add.block, &machine->l3);
    
    
        Line* cache1 = machine->l1.lines;
        Line* cache2 = machine->l2.lines;
        Line* cache3 = machine->l3.lines;
    
        MemoryBlock* RAM = machine->ram.blocks;
    
        if (cache1[l1pos].tag == add.block) { 
            cache1[l1pos].cost = COST_ACCESS_L1;
            cache1[l1pos].cacheHit = 1;
        } else if (cache2[l2pos].tag == add.block) { 
            /* Block is in memory cache L2 */
            cache2[l2pos].tag = add.block;
            cache2[l2pos].updated = false;
            cache2[l2pos].cost = COST_ACCESS_L1 + COST_ACCESS_L2;
            cache2[l2pos].cacheHit = 2;
    
            updateMachineInfos(machine, &(cache2[l2pos]));
    
            return &(cache2[l2pos]);
        } else if (cache3[l3pos].tag == add.block){
            /* Block is in memory cache L3 */
            cache3[l3pos].tag = add.block;
            cache3[l3pos].updated = false;
            cache3[l3pos].cost = COST_ACCESS_L1 + COST_ACCESS_L2 + COST_ACCESS_L3;
            cache3[l3pos].cacheHit = 3;
    
            updateMachineInfos(machine, &(cache3[l3pos]));
            return &(cache3[l3pos]);
    
        } else { 
            /* Block only in memory RAM, need to bring it to cache and manipulate the blocks */
            //fazer o mapeamento para decidir quem tirar da ram.
            
            if (!canOnlyReplaceBlock(cache1[l1pos])) { 
                /* The block on cache L1 cannot only be replaced, the memories must be updated */
                if (!canOnlyReplaceBlock(cache2[l2pos])){
                    /* The block on cache L2 cannot only be replaced, the memories must be updated */
                    if(!canOnlyReplaceBlock(cache3[l3pos])){
                        /* The block on cache L2 cannot only be replaced, the memories must be updated */
                        RAM[cache3[l3pos].tag] = cache3[l3pos].block;
                    }
                    cache3[l3pos] = cache2[l2pos];
                    cache3[l3pos].timeOnCache = 0;
    
    
                }
    
                cache2[l2pos] = cache1[l1pos];
                cache2[l2pos].timeOnCache = 0;
    
            }
    
            
            cache1[l1pos].block = RAM[add.block];
            cache1[l1pos].timesUsed = 1;
            cache1[l1pos].tag = add.block;
            cache1[l1pos].updated = false;
            cache1[l1pos].cost = COST_ACCESS_L1 + COST_ACCESS_L2 + COST_ACCESS_L3 + COST_ACCESS_RAM;
            cache1[l1pos].timeOnCache = 0;
            cache1[l1pos].cacheHit = 4;
    
    
    
        }
        updateMachineInfos(machine, &(cache1[l1pos]));
        return &(cache1[l1pos]);
    }

\end{lstlisting}

\clearpage

\subsection{Função Main}

Na função main adicionamos mais um espaço a memoriesSize para o HD

\begin{lstlisting}[caption={Main},label={lst:cod6},language=C]
    int main(int argc, char**argv) {

        srand(1507);   // Inicializacao da semente para os numeros aleatorios.

        if (argc != 6) {
            printf("Numero de argumentos invalidos! Sao 6.\n");
            printf("Linha de execucao: ./exe TIPO_INSTRUCAO [TAMANHO_RAM|ARQUIVO_DE_INSTRUCOES] TAMANHO_L1 TAMANHO_L2 TAMANHO_L3\n");
            printf("\tExemplo 1 de execucao: ./exe random 10 2 4 6\n");
            printf("\tExemplo 2 de execucao: ./exe file arquivo_de_instrucoes.txt\n");
            return 0;
        }

        int memoriesSize[4];
        Machine machine;
        Instruction *instructions;

        memoriesSize[1] = atoi(argv[3]);
        memoriesSize[2] = atoi(argv[4]);
        memoriesSize[3] = atoi(argv[5]);


        if (strcmp(argv[1], "random") == 0) {
            memoriesSize[0] = atoi(argv[2]);
            instructions = generateRandomInstructions(memoriesSize[0]);
        } else if (strcmp(argv[1], "file") == 0) {
            instructions = readInstructions(argv[2], memoriesSize);
        } 
        else {
            printf("Invalid option.\n");
            return 0;
        }
        
        printf("Starting machine...\n");
        start(&machine, instructions, memoriesSize);
        if (memoriesSize[0] < 10)
            printMemories(&machine);
        run(&machine);
        if (memoriesSize[0] < 10)
            printMemories(&machine);
        stop(&machine);
        printf("Stopping machine...\n");
        return 0;
    }
\end{lstlisting}
\clearpage


\section{Impressões Gerais}
Primeiramente, nos reunimos no discord para a leitura e compreensão do documento disponibilizado para a realização do trabalho.
Foi bem difícil compreender como era pra ser feito esse trabalho, a parte de realizar a interrupção foi ainda mais díficil pois não tivemos conteúdo o 
suficiente para nos basear e "ter um norte" para começar o trabalho. A simulação de um HD também trouxe dificuldades, faltou uma ideia boa de como ocorreria essa implementação, 
e acabamos ficando perdidos sobre a implementação. Vale lembrar que infelizmente faltou um pouco de tempo, considerando que esse periodo da UFOP foi mais corrido que o normal,
havendo uma sobrecarga de atividades. Por isso devido a complexidade das funções a serem implementadas e pela falta de material para pesquisa acabamos deixando a desejar em alguns aspectos como
a implementação do MMUSearchOnMemorys com o HD.
\clearpage
\section{Análise}

Tivemos uma ideia boa de como deveria ocorrer a implementação do HD, chegamos a realizar a montagem inicial do TAD e dos parâmetros necessários para a implementação
do HD. Além disso criamos os arquivos para simular o acesso em memória externa, assim como ter uma ideia da lentidão da memória externa(comumente conhecida como SWAP em sistemas UNIX),
juntamente com um tratador de interrupções, que se prova extremamente util, especialmente em sistemas operacionais, porque sempre haverá algumas tarefas com prioridades maiores a serem executadas.

\subsection{Análise de Complexidade}
Na função memoryCacheMapping, a sua complexidade vai depender do método de mapeamento escolhido. Para o mapeamento direto, 
como não percorremos o vetor, já que a posição na cache depende unicamente do final de seu endereço, 
sua complexidade será $\mathcal{O}(1)$. \\

Na política LRU, na função memoryCacheMapping, percorremos o vetor para encontrar o bloco que está a mais tempo sem ser utilizado, por isso 
teremos uma complexidade de $\mathcal{O}(n)$. \\

Na política LFU, na memoryCacheMapping, também percorremos o vetor para encontrar o bloco que está a mais tempo sem ser utilizado, por isso 
teremos uma complexidade de $\mathcal{O}(n)$.

\clearpage
\section{Conclusão}
Dentro do possível o trabalho apresentou uma didática interessantes, nos fornecendo um jeito mais palpável
de aprender conceitos intrisicos a um sistema operacional. Porem algumas coisas deixaram a desejar, sendo a principal o material fornecido para 
podermos realizar o trabalho e o prazo, que entendemos que não seja culpa apenas do professor, mas também do calendário acadêmico. Com isso a parte que conseguimos
finalizar do trabalho nos deu uma excelente ideia do funcionamento e como computadores são organizados desde o conceito mais básico, como uma simples instrução de soma, até
ao conceito de interupções e armazenamento, tanto interno quanto externo. Sendo uma disciplina extremamente proveitosa e que irá agregar para a nossa bagagem como profissionais.


\end{document}